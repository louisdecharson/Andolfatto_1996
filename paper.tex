\documentclass[english]{article}
\usepackage[T1]{fontenc}
\usepackage[latin9]{inputenc}
\usepackage{geometry}
\geometry{verbose,tmargin=2cm,bmargin=2cm,lmargin=2cm,rmargin=2cm}
\usepackage{calc}
\usepackage{babel}
\usepackage{amsmath}
\begin{document}

\title{Business cycles and labor market search (David Andolfato)}


\author{Louis and Monica}

\maketitle

\section{The model}

The model nest a labor-market-search framework into an standard RBC
model. He assume that there are households distributed uniformly on
the unit interval and that they have preferences for consumption and
leisure. In the model, households face the standard consumption-saving
problem, but face altogether different opportunities for exchanging
labor services. Individuals either have a job opportunity or not,
and job opportunities come and go at random, depending to some extent
on individual search effort, the availability of jobs, and plain luck.
Firms also face a standard wealth maximization problem, except that,
because finding new workers takes time and effort, firms view their
existing workforce as a capital asset. Given constant returns in the
production technology, it may be assumed without loss that each firm
comprises a single job; in what follows, the terms firm and job will
be used interchangeably.


\subsection{The search process}

In order to produce output, each job requires a worker. Let $n_{t}$,
denote the number of jobs that are matched with a worker at the beginning
of period $t$. Job-worker pairs are assumed to separate at the exogenous
rate $0<\sigma<1$. Replenishing this stock takes time and consumes
resources. 

A firm interested in filling an available job must undertake recruiting
and screening activities, which are necessary for finding a suitable
employee. Let $v_{t}$, denote the number of jobs vacancies during
period $t$, each of which incurs a flow cost equal to $\kappa>0$,
measured in units of physical output. In the version of the model
studied here, workers are assumed to search passively. Letting $e$
denote search effort per worker seeking employment, aggregate search
effort by workers is given by $(1-n_{t})e$. The rate at which new
job matches form is governed by an aggregate matching technology,
$M(v,(1-n)e)$, so that employment evolves according to the following
dynamic equation: 

\begin{equation}
n_{t+1}=(1-\sigma)n_{t}+M(v_{t},(1-n_{t})e)\label{eq:dynamic of n}
\end{equation}


The matching technology is assumed to be a non-decreasing, concave
function of aggregate search and recruiting effort and is assumed
to display constant returns to scale.


\subsection{The Social Welfare Problem}

The representative household has preferences represented by a utility
function of the following form:

\begin{equation}
E_{0}\sum_{t=0}^{\infty}\beta^{t}\left[U(c_{t})+\phi(t)H(1-x_{t})\right]\label{eq:utility function}
\end{equation}


where $c_{t}$ denotes consumption, $x_{t}$, denotes the fraction
of time spent in nonleisure activities and $0<\beta<1$ is a discount
factor. The functions $U$ and $H$ are increasing and concave. The
value of the parameter $\phi(t)>0$ depends on a household's employment
status: $\phi(t)$ is equal to $\phi_{1}$ if the household is employed
and is equal to $\phi_{2}$ if the household is unemployed.

Output is produced according to a standard neoclassical production
technology, $y_{t}=F(k,\: n_{t}l_{t};\: z_{t})$, where $k_{t}$ is
the aggregate capital input; $l_{t}$ is average hours worked by those
employed; and $z_{t}$ is a parameter reflecting the current state
of technology, which evolves stochastically according to the transition
function $G(z',z)=Pr[z_{t+1}\leq z'\mid z_{t}=z]$. The capital stock
depreciates at rate $0<\delta<1$, so that the economy-wide resource
constraint is then given by:

\begin{equation}
c_{t}+k_{t+1}+\kappa v_{t}=y_{t}+(1-\delta)k_{t}\label{eq:budget constraint}
\end{equation}


$s\equiv(k,\: n\:,z)$ denote the vector of the state variables of
the economic system, $E_{G}$ denote the expectation operator associated
with the transition function $G$, $W(s_{0})$ denote the maximum
value of (\ref{eq:utility function}), given an arbitrary initial
condition ($s_{0}$), which is obtained by solving the welfare problem
stated above. The primed variables denote 'next period values'.


\subsection{Equations of the model to log-linearize}

\begin{equation} \label{eq:P1}
U_{c}(c)=\beta E_{G}\left[F_{k}(k',n'l';\: z')+(1-\delta)\right]U_{c}(c')
\end{equation}


\begin{equation} \label{eq:P2}
\phi_{1}H_{l}(1-l)=F_{l}(k,\: nl;\: z)U_{c}(c)
\end{equation}
 

\begin{equation} \label{eq:P3}
\kappa vU_{c}(c)=\mu\alpha M(v,\:(1-n)e)
\end{equation}


\begin{equation} \label{eq:P4}
\mu=\beta E_{G}\left\{ \phi_{1}H(1-l')-\phi_{2}H(1-e)+U_{c}(c')F_{n'}(k',n'l';\: z')l'+\mu'\left[1-\sigma-(1-\alpha)p(v',n')\right]\right\} 
\end{equation}


\begin{equation} \label{eq:P5}
c+k'+\kappa v=F(k,\: nl;\: z)+(1-\delta)k
\end{equation}


\begin{equation} \label{eq:P6}
n'=(1-\sigma)n+M(v,\:(1-n)e)
\end{equation}



\begin{equation} \label{eq:P7}
w=(1-\alpha)F_{2}(k,\: nl;\: z)+\alpha\left[\frac{\left[\phi_{2}H(1-e)-\phi_{1}H(1-l)\right]+p(v,\: n)(1-\alpha)\mu}{U_{c}(c)}\right]\frac{1}{l}
\end{equation}


The functional forms for $U$, $H$, $F$, $M$ and $G$ are as follows:
\begin{itemize}
\item $U(c)=log(c)$
\item $H(1-x)=\frac{\left(1-x\right)^{(1-\eta)}}{(1-\eta)}$
\item $F(k,\: nl;\: z)=exp(z)\zeta k^{\theta}(nl)^{(1-\theta)}$
\item $M(v,\:(1-n)e)=min\left\{ v,\:1-n,\:\chi v^{\alpha}((1-n)e)^{(1-\alpha)}\right\} $
\item $p(v,\: n)=\frac{\chi v^{\alpha}((1-n)e)^{(1-\alpha)}}{1-n}$
\end{itemize}

Additionally, the productivity shock is assumed to be governed by
the following stochastic process:
\begin{itemize}
\item $z'=\rho z+\tilde{\varepsilon}$
\end{itemize}
where $0<\rho<1$ and $\tilde{\varepsilon}$ is an independently and identically distributed random variable. In particular, assume that
$\tilde{\varepsilon}\in\left\{ -\varepsilon,\:\varepsilon\right\} $,$\varepsilon>0$, and $prob(\varepsilon)$ = $prob(-\varepsilon)$=$\frac{1}{2}$ . \\
Then equations (\ref{eq:P1}) to (\ref{eq:P7}) take the form:


\begin{itemize}
\item (\ref{eq:P1}') $\frac{1}{c}=\beta E_{G}\left[\theta exp(z')\zeta(\frac{n'l'}{k'})^{(1-\theta)}+(1-\delta)\right]\frac{1}{c'}$
\item (\ref{eq:P2}') $\phi_{1}\left(1-l\right)^{-\eta}=(1-\theta)exp(z)\zeta k^{\theta}(nl)^{-\theta}\frac{1}{c}$
\item (\ref{eq:P3}') $\kappa v\frac{1}{c}=\mu\alpha\chi v^{\alpha}((1-n)e)^{(1-\alpha)}$
\item (\ref{eq:P4}') $\mu=\beta E_{G}\left\{ \phi_{1}\frac{\left(1-l'\right)^{(1-\eta)}}{(1-\eta)}-\phi_{2}\frac{\left(1-e\right)^{(1-\eta)}}{(1-\eta)}+\frac{1}{c'}(1-\theta)exp(z)\zeta k'^{\theta}(n'l')^{-\theta}l'+\mu'\left[1-\sigma-(1-\alpha)\frac{\chi v'^{\alpha}((1-n')e)^{(1-\alpha)}}{1-n'}\right]\right\} $
\item (\ref{eq:P5}') $c+k'+\kappa v=exp(z)\zeta k^{\theta}(nl)^{(1-\theta)}+(1-\delta)k$
\item (\ref{eq:P6}') $n'=(1-\sigma)n+\chi v^{\alpha}((1-n)e)^{(1-\alpha)}$
\item {\small (\ref{eq:P7}') $w=(1-\alpha)(1-\theta)exp(z)\zeta k^{\theta}(nl)^{-\theta}+\alpha\left[\left(\phi_{2}\frac{\left(1-e\right)^{(1-\eta)}}{(1-\eta)}-\phi_{1}\frac{\left(1-l\right)^{(1-\eta)}}{(1-\eta)}\right)+(1-\alpha)\frac{\chi v{}^{\alpha}((1-n)e)^{(1-\alpha)}}{1-n}\mu\right]\frac{c}{l}$}{\small \par}
\end{itemize}

\subsection{Equations at the steady state}
\begin{itemize}
\item (\ref{eq:P1}'') $1=\beta\left[\theta\zeta(\frac{n*l*}{k*})^{(1-\theta)}+(1-\delta)\right]$
\item (\ref{eq:P2}'') $\phi_{1}\left(1-l*\right)^{-\eta}=(1-\theta)\zeta(k*)^{\theta}(n*l*)^{-\theta}\frac{1}{c*}$
\item (\ref{eq:P3}'') $\frac{\kappa v*}{c*}=\mu*\alpha\chi(v*)^{\alpha}((1-n*)e)^{(1-\alpha)}$
\item (\ref{eq:P4}'') $\mu*=\beta\left\{ \phi_{1}\frac{\left(1-l*\right)^{(1-\eta)}}{(1-\eta)}-\phi_{2}\frac{\left(1-e\right)^{(1-\eta)}}{(1-\eta)}+(1-\theta)\zeta(k*)^{\theta}(n*l*)^{-\theta}\frac{l*}{c*}+\mu*\left[1-\sigma-(1-\alpha)\frac{\chi(v*)^{\alpha}((1-n*)e)^{(1-\alpha)}}{1-n*}\right]\right\} $
\item (\ref{eq:P5}'') $c*+\delta k*+\kappa v*=\zeta(k*)^{\theta}(n*l*)^{(1-\theta)}$
\item (\ref{eq:P6}'') $\sigma n*=\chi(v*)^{\alpha}((1-n*)e)^{(1-\alpha)}$
\item (\ref{eq:P7}'') $w*n*l*=(1-\theta)-\frac{\kappa v*}{\sigma}\left[\left(\frac{1-\left(1-\sigma\right)\beta}{\beta}\right)\right]$
\end{itemize}

\subsection{Log-linearization around the steady state}

we denote $\frac{\partial X_{t}}{X*}=\widetilde{X}_{t}$, and $X$ the value of $X$ at the steady state while $X_{t}$ is the value of $X$ at time $t$. \\

\subsubsection{Equation (\ref{eq:P1})'}

From (\ref{eq:P1}') and (\ref{eq:P7}'') we have :
\begin{align*}
\cfrac{1}{c_t} &= \beta E_{t}\left\{\theta exp(z_{t+1})\zeta\left(\frac{n_{t+1}l_{t+1}}{k_{t+1}}\right)^{(1-\theta)}+(1-\delta)\right\}\frac{1}{c_{t+1}} \\
\end{align*}
\begin{multline*}
-\cfrac{\tilde{c_t}}{c} = \beta E_{t}\left\{\theta\zeta(\frac{nl}{k})^{(1-\theta)}\widetilde{z_{t+1}}+(1-\theta)\theta\zeta(\frac{nl}{k})^{-\theta}\frac{ln}{k}\frac{\partial n_{t+1}}{n}+(1-\theta)\theta\zeta(\frac{nl}{k})^{-\theta}\frac{ln}{k}\frac{\partial l_{t+1}}{l}+(1-\theta)\theta\zeta(\frac{nl}{k})^{-\theta}\left(nl\frac{-\partial k_{t+1}}{k^{2}}\right)\right\}\frac{1}{c}\\+\beta E_{t}\left\{\left(\zeta\theta\left(\frac{nl}{k}\right)^{(1-\theta)}+1-\delta\right)(-\frac{\partial c_{t+1}}{c^2})\right\}
\end{multline*}
\begin{align*}
-\widetilde{c_t} &= \beta E_{t}\left\{\zeta*\theta*\left(\frac{nl}{k}\right)^{(1-\theta)}\left((1-\theta)(\widetilde{l_{t+1}}+\widetilde{n_{t+1}}-\widetilde{k_{t+1}})+\widetilde{z_{t+1}}\right)\right\}-\underbrace{\beta\left(\zeta*\theta*\left(\frac{nl}{k}\right)^{(1-\theta)}+1-\delta\right)}_{=1}E_t\widetilde{c_{t+1}}\\
\widetilde{c_t} &= E_t\widetilde{c_{t+1}}-\beta E_{t}\left\{\zeta*\theta*\left(\frac{nl}{k}\right)^{(1-\theta)}\left((1-\theta)(\widetilde{l_{t+1}}+\widetilde{n_{t+1}}-\widetilde{k_{t+1}})+\widetilde{z_{t+1}}\right)\right\}
\end{align*}

\subsubsection{Equation (\ref{eq:P2})'}
\begin{align*}
  \phi_{1}\left(1-l\right)^{-\eta} &= (1-\theta)exp(z)\zeta k^{\theta}(nl)^{-\theta}\frac{1}{c} \\
\end{align*}
\begin{multline*}
  -\eta\phi_{1}\left(1-l\right)^{-\eta-1}(\frac{-\partial l_{t}l}{l}) = (1-\theta)\zeta k^{\theta}(nl)^{-\theta}\frac{1}{c}\widetilde{z_{t}}+\theta(1-\theta)\zeta k^{\theta-1}(nl)^{-\theta}\frac{1}{c}\partial k_{t}-\theta(1-\theta)\zeta k^{\theta}(nl)^{-\theta}\frac{1}{c}\left(\partial n_{t}n+\partial l_{t}n\right)\\-(1-\theta)\zeta k^{\theta}(nl)^{-\theta}\frac{\partial c_{t}}{c^{2}}
\end{multline*}
\begin{align*}
\eta\phi_{1}\left(1-l\right)^{-\eta-1}\widetilde{l}_{t}l &=(1-\theta)\zeta k^{\theta}(nl)^{-\theta}\frac{1}{c}\left[\widetilde{z_{t}}+\theta\left(\widetilde{k}_{t}-\widetilde{n}_{t}-\widetilde{l}_{t}\right)-\widetilde{c}_{t}\right]\\
  \frac{l\eta}{1-l}\widetilde{l}_{t}&=\left[\widetilde{c}_{t}-\widetilde{z_{t}}+\theta\left(\widetilde{n}_{t}+\widetilde{l}_{t}-\widetilde{k}_{t}\right)\right]
\end{align*}

\subsubsection{Equation (\ref{eq:P3})'}

\begin{align*}
  \kappa v\frac{1}{c}&=\mu\alpha\chi v^{\alpha}((1-n)e)^{(1-\alpha)} \\
  \kappa v\frac{1}{c}\left(\widetilde{v}_{t}-\widetilde{c}_{t}\right) &=\alpha\mu\alpha\chi v^{\alpha}((1-n)e)^{-\alpha}(\widetilde{\mu}_{t}+\widetilde{v}_{t})+(1-\alpha)\mu\alpha\chi v^{\alpha}((1-n)e)^{-\alpha}(-\frac{\partial n_{t}ne}{n}) \\
  \kappa v\frac{1}{c}\left(\widetilde{v}_{t}-\widetilde{c}_{t}\right) &=\alpha\mu\alpha\chi v^{\alpha}((1-n)e)^{1-\alpha}\left[\widetilde{\mu}_{t}+\alpha\widetilde{v}_{t}-(1-\alpha)\frac{ne}{(1-n)e}\widetilde{n}_{t}\right]\\
  (1-\alpha)\widetilde{v}_{t}-\widetilde{c}_{t}&=\widetilde{\mu}_{t}-(1-\alpha)\frac{n}{(1-n)}\widetilde{n}_{t} 
\end{align*}



\subsubsection{Equation (\ref{eq:P4})'}
\begin{align*}
  \mu &= \beta E_{G}\left\{ \phi_{1}\frac{\left(1-l'\right)^{(1-\eta)}}{(1-\eta)}-\phi_{2}\frac{\left(1-e\right)^{(1-\eta)}}{(1-\eta)}+\frac{1}{c'}(1-\theta)exp(z')\zeta k'^{\theta}(n'l')^{-\theta}l'+\mu'\left[1-\sigma-(1-\alpha)\frac{\chi v'^{\alpha}((1-n')e)^{(1-\alpha)}}{1-n'}\right]\right\}\\
\end{align*}
\begin{multline*}
  \widetilde{\mu}_{t}\mu=\beta\left\{ -(1-\eta)\phi_{1}\frac{\left(1-l\right)^{(-\eta)}}{(1-\eta)}\widetilde{l}_{t+1}l+\frac{1}{c}(1-\theta)\zeta k^{\theta}(nl)^{-\theta}lE_{t}\left[\widetilde{z}_{t+1}+\theta\left(\widetilde{k}_{t+1}-\widetilde{n}_{t+1}-\widetilde{l}_{t+1}\right)-\widetilde{c}_{t+1}\right]\right.\\
+E_{t}\widetilde{\mu}_{t+1}\left[1-\sigma-(1-\alpha)\frac{\chi v\alpha((1-n)e)^{(1-\alpha)}}{1-n}\right]-\mu(1-\alpha)\frac{\chi v\alpha((1-n)e)^{(1-\alpha)}}{1-n}\left(\alpha E_{t}\widetilde{v}_{t+1}\right)\\
\left. -\mu(-\alpha)(1-\alpha)\chi v^{\alpha}e^{1-\alpha}(-\widetilde{n_{t+1}}n)*(1-n)^{-\alpha-1}\right\}
\end{multline*}
\begin{multline*}
  \widetilde{\mu}_{t}\mu=\beta\left\{\phi_{1}\frac{l}{(1-l)^{eta}}E_t\left(\tilde{z_{t+1}}+\theta(\tilde{k_{t+1}}-\tilde{n_{t+1}}-\tilde{l_{t+1}})-\tilde{c_{t+1}}\right)+\left(1-\sigma-(1-\alpha)\frac{\sigma n}{1-n}\right)E_t\tilde{\mu_{t+1}}\right.\\
\left. -\frac{\sigma n\alpha(1-\alpha)}{1-n}E_t\right(\tilde{v_{t+1}}+\frac{\tilde{n_{t+1}}n}{1-n}\left) \right\} 
\end{multline*}


\subsubsection{Equation (\ref{eq:P5})'}
\begin{align*}
  c+k'+\kappa v&=exp(z)\zeta k^{\theta}(nl)^{(1-\theta)}+(1-\delta)k\\
  c\widetilde{c}_{t}+k\widetilde{k}_{t+1}+\kappa v\widetilde{v}_{t}&=\zeta k^{\theta}(nl)^{(1-\theta)}\left(\widetilde{z}_{t}+\theta\widetilde{k}_{t}+(1-\theta)(\widetilde{n}_{t}+\widetilde{l}_{t})\right)+(1-\delta)k\widetilde{k}_{t}\\
c\widetilde{c}_{t}+k\widetilde{k}_{t+1}+\kappa v\widetilde{v}_{t}&=\zeta k^{\theta}(nl)^{(1-\theta)}\left(\widetilde{z}_{t}+\theta\widetilde{k}_{t}+(1-\theta)(\widetilde{n}_{t}+\widetilde{l}_{t})\right)+(1-\delta)k\widetilde{k}_{t}\\
\end{align*}

\subsubsection{Equation (\ref{eq:P6})'}
\begin{align*}
  n'&=(1-\sigma)n+\chi v^{\alpha}((1-n)e)^{(1-\alpha)} \\
  n\widetilde{n}_{t+1}&=(1-\sigma)n\widetilde{n}_{t}+\sigma n\left[\alpha\widetilde{v}_{t}-(1-\alpha)\frac{n}{(1-n)}\widetilde{n}_{t}\right]\\
  \widetilde{n}_{t+1}&=(1-\sigma)\widetilde{n}_{t}+\sigma\left[\alpha\widetilde{v}_{t}-(1-\alpha)\frac{n}{(1-n)}\widetilde{n}_{t}\right] \\
  \widetilde{n}_{t+1}&=\sigma\alpha\widetilde{v}_{t}+\frac{(1-n)-\sigma(1-\alpha n)}{(1-n)}\widetilde{n}_{t}\\
\end{align*}

\subsubsection{Equation (\ref{eq:P7})'}
\begin{align*}
w &=(1-\alpha)(1-\theta)exp(z)\zeta k^{\theta}(nl)^{-\theta}+\alpha\left[\left(\phi_{2}\frac{\left(1-e\right)^{(1-\eta)}}{(1-\eta)}-\phi_{1}\frac{\left(1-l\right)^{(1-\eta)}}{(1-\eta)}\right)+(1-\alpha)\frac{\chi v{}^{\alpha}((1-n)e)^{(1-\alpha)}}{1-n}\mu\right]\frac{c}{l} \\
\end{align*}
\begin{multline*}
w\widetilde{w}_{t}=(1-\alpha)(1-\theta)\zeta k^{\theta}(nl)^{-\theta}\left[\widetilde{z}_{t}+\theta\left(\widetilde{k}_{t}-\widetilde{n}_{t}-\widetilde{l}_{t}\right)\right]\\
+\alpha\left[\phi_{1}\left(1-l\right)^{-\eta}\widetilde{l}_{t}l+(1-\alpha)\frac{\chi v{}^{\alpha}((1-n)e)^{(1-\alpha)}}{1-n}\mu\left[\mu\widetilde{\mu}_{t}+\alpha\widetilde{v}_{t}\right]\right]\frac{c}{l}\\
+\alpha\left[\left(\phi_{2}\frac{\left(1-e\right)^{(1-\eta)}}{(1-\eta)}-\phi_{1}\frac{\left(1-l\right)^{(1-\eta)}}{(1-\eta)}\right)+(1-\alpha)\frac{\chi v{}^{\alpha}((1-n)e)^{(1-\alpha)}}{1-n}\mu\right]\frac{c}{l}\left(\widetilde{c}_{t}-\widetilde{l}_{t}\right)  
\end{multline*}
\begin{multline*}
w\widetilde{w}_{t}=(1-\alpha)\phi_{1}\left(1-l\right)^{-\eta}c\left[\widetilde{z}_{t}+\theta\left(\widetilde{k}_{t}-\widetilde{n}_{t}-\widetilde{l}_{t}\right)\right]\\
+\alpha\left[\phi_{1}\left(1-l\right)^{-\eta}\widetilde{l}_{t}l+\frac{(1-\alpha)\sigma n}{\left(1-n\right)}\mu\left[\widetilde{\mu}_{t}+\alpha\widetilde{v}_{t}\right]\right]\frac{c}{l}\\
+\alpha\left[\mu\left((1-\sigma)-\frac{1}{\beta}\right)+\phi_{1}\left(1-l\right)^{-\eta}\right]\frac{c}{l}\left(\widetilde{c}_{t}-\widetilde{l}_{t}\right)  
\end{multline*}

\begin{multline*}
  w\widetilde{w}_{t}=(1-\alpha)\phi_{1}\left(1-l\right)^{-\eta}c\left[\widetilde{z}_{t}+\theta\left(\widetilde{k}_{t}-\widetilde{n}_{t}-\widetilde{l}_{t}\right)\right]\\
+\alpha\left[\phi_{1}\left(1-l\right)^{-\eta}\widetilde{l}_{t}l+\frac{(1-\alpha)\sigma n}{\left(1-n\right)}\mu\left[\widetilde{\mu}_{t}+\alpha\widetilde{v}_{t}\right]\right]\frac{c}{l}\\
+\alpha\left[\mu\left((1-\sigma)-\frac{1}{\beta}\right)+\phi_{1}\left(1-l\right)^{-\eta}\right]\frac{c}{l}\left(\widetilde{c}_{t}-\widetilde{l}_{t}\right)
\end{multline*}


\subsection{Matrix Form}
The model has eight variables including three state variables $(k,n,z)$. 
We start by the three static equations (\ref{eq:P2}),(\ref{eq:P3}) and (\ref{eq:P7}). We write them in matrix form, with $M_1$ \& $M_2$ $3x5$ matrixes.
\begin{align}
M_1
\begin{pmatrix}
  \widetilde{c_t} \\
  \widetilde{l_t} \\
  \widetilde{v_t} \\
  \widetilde{\mu_t} \\
  \widetilde{w_t} \\
\end{pmatrix}
&= M_2
\begin{pmatrix}
  \widetilde{k_t}\\
  \widetilde{n_t}\\
  \widetilde{z_t}\\
\end{pmatrix}
\end{align}
We have five dynamic equations, we write them in matrix form, with $M_3^{I}$, $M_3^{L}$ $5x3$ matrices, $M_{4}^{L}$, $M_{4}^{I}$ $5x5$ matrices and $M_{5}$ a $5x6$ :
\begin{align}
  M_3^{I}\begin{pmatrix}
    \widetilde{k_{t+1}} \\
    \widetilde{n_{t+1}} \\
    \widetilde{z_{t+1}} \\
  \end{pmatrix} +
M_3^{L}\begin{pmatrix}
  \widetilde{k_{t}} \\
  \widetilde{n_{t}} \\
  \widetilde{z_{t}} \\
\end{pmatrix}
&= M_4^{I}\begin{pmatrix}
  \widetilde{c_{t+1}} \\
  \widetilde{l_{t+1}} \\
  \widetilde{v_{t+1}} \\
  \widetilde{\mu_{t+1}} \\
  \widetilde{w_{t+1}} \\
\end{pmatrix} 
+ M_4^{L}\begin{pmatrix}
  \widetilde{c_{t}} \\
  \widetilde{l_{t}} \\
  \widetilde{v_{t}} \\
  \widetilde{\mu_{t}} \\
  \widetilde{w_{t}} \\
\end{pmatrix}
+M_5 \begin{pmatrix}
  \widetilde{\epsilon^{k}}_{t+1} \\
  \widetilde{\epsilon^{n}}_{t+1} \\
  \widetilde{\epsilon^{z}}_{t+1} \\
  \widetilde{\epsilon^{c}}_{t+1} \\
  \widetilde{\epsilon^{l}}_{t+1} \\
  \widetilde{\epsilon^{v}}_{t+1} \\
\end{pmatrix}
\end{align}



\newpage

\section*{Annex}
\label{Annex}
\subsection*{Wage equation at the steady state}
\label{subsec:annexI}

In this section, we prove that the wage equation at the steady state equals (\ref{eq:P7}'').

\begin{align*}
w*&=(1-\alpha)(1-\theta)\zeta(k*)^{\theta}(n*l*)^{-\theta}+\alpha\left[\left(\phi_{2}\frac{\left(1-e\right)^{(1-\eta)}}{(1-\eta)}-\phi_{1}\frac{\left(1-l*\right)^{(1-\eta)}}{(1-\eta)}\right)-(1-\alpha)\frac{\chi(v*)^{\alpha}((1-n*)e)^{(1-\alpha)}}{1-n}\mu*\right]\frac{c*}{l*} \\
w*&=(1-\alpha)(1-\theta)\zeta(k*)^{\theta}(n*l*)^{-\theta}+\alpha\left[\left(\phi_{2}\frac{\left(1-e\right)^{(1-\eta)}}{(1-\eta)}-\phi_{1}\frac{\left(1-l*\right)^{(1-\eta)}}{(1-\eta)}\right)-(1-\alpha)\frac{\chi(v*)^{\alpha}((1-n*)e)^{(1-\alpha)}}{1-n*}\mu*\right]\frac{c*}{l*} \\
\end{align*} 

From (\ref{eq:P4}'') we have 

\begin{multline*}
 \frac{\mu*}{\beta}-\mu*(1-\sigma) = \phi_{1}\frac{\left(1-l*\right)^{(1-\eta)}}{(1-\eta)}-\phi_{2}\frac{\left(1-e\right)^{(1-\eta)}}{(1-\eta)} \\+\frac{1}{c*}(1-\theta)\zeta(k*)^{\theta}(n*l*)^{-\theta}l*+\mu*\left[-(1-\alpha)\frac{\chi(v*)^{\alpha}((1-n*)e)^{(1-\alpha)}}{1-n*}\right]
\end{multline*}
\begin{multline*}
\mu*(1-\sigma)-\frac{\mu*}{\beta}+\frac{1}{c*}(1-\theta)\zeta(k*)^{\theta}(n*l*)^{-\theta}l* = \phi_{2}\frac{\left(1-e\right)^{(1-\eta)}}{(1-\eta)}-\phi_{1}\frac{\left(1-l*\right)^{(1-\eta)}}{(1-\eta)}\\ -\mu*\left[(1-\alpha)\frac{\chi(v*)^{\alpha}((1-n*)e)^{(1-\alpha)}}{1-n*}\right]
\end{multline*}
\begin{multline*}
\mu*\left((1-\sigma)-\frac{1}{\beta}\right)+\frac{1}{c*}(1-\theta)\zeta(k*)^{\theta}(n*l*)^{-\theta}l* = \phi_{2}\frac{\left(1-e\right)^{(1-\eta)}}{(1-\eta)}-\phi_{1}\frac{\left(1-l*\right)^{(1-\eta)}}{(1-\eta)}\\ -\mu*\left[(1-\alpha)\frac{\chi(v*)^{\alpha}((1-n*)e)^{(1-\alpha)}}{1-n*}\right]
\end{multline*}

Replacing in the previous eq. we have

\begin{align*}
w* &=(1-\alpha)(1-\theta)\zeta(k*)^{\theta}(n*l*)^{-\theta}+\alpha\left[\mu*\left((1-\sigma)-\frac{1}{\beta}\right)+\frac{1}{c*}(1-\theta)\zeta(k*)^{\theta}(n*l*)^{-\theta}l*\right]\frac{c*}{l*} \\
w* &=(1-\alpha)(1-\theta)\zeta(k*)^{\theta}(n*l*)^{-\theta}+\alpha(1-\theta)\zeta(k*)^{\theta}(n*l*)^{-\theta}+\alpha\left[\mu*\left((1-\sigma)-\frac{1}{\beta}\right)\right]\frac{c*}{l*} \\
w* &=(1-\theta)\zeta(k*)^{\theta}(n*l*)^{-\theta}+\alpha\left[\mu*\left(\frac{\left(1-\sigma\right)\beta-1}{\beta}\right)\right]\frac{c*}{l*} \\
w*n*l* &=(1-\theta)\zeta(k*)^{\theta}(n*l*)^{1-\theta}+\alpha\left[\mu*\left(\frac{\left(1-\sigma\right)\beta-1}{\beta}\right)\right]c*n*
\end{align*}

From :
\begin{align*}
 \frac{\kappa v*}{c*} &= \mu*\alpha\chi(v*)^{\alpha}((1-n*)e)^{(1-\alpha)} 
\end{align*}

We have :
\begin{align*}
w*n*l* &=(1-\theta)\zeta(k*)^{\theta}(n*l*)^{1-\theta}+\frac{\kappa v*}{c*\left[\chi(v*)^{\alpha}((1-n*)e)^{(1-\alpha)})\right]}\left[\left(\frac{\left(1-\sigma\right)\beta-1}{\beta}\right)\right]c*n* \\
w*n*l* &=(1-\theta)\zeta(k*)^{\theta}(n*l*)^{1-\theta}+\frac{\kappa v*}{\left[\sigma n*)\right]}\left[\left(\frac{\left(1-\sigma\right)\beta-1}{\beta}\right)\right]n* \\
w*n*l* &=(1-\theta)\zeta(k*)^{\theta}(n*l*)^{1-\theta}-\frac{\kappa v*}{\sigma}\left[\left(\frac{1-\left(1-\sigma\right)\beta}{\beta}\right)\right]
\end{align*}

With the technology parameter $\zeta$ chosen to normalize the steady-state level of output to unity, labor share can be computed as :
\begin{align*}
 w*n*l*=(1-\theta)-\frac{\kappa v*}{\sigma}\left[\left(\frac{1-\left(1-\sigma\right)\beta}{\beta}\right)\right]
\end{align*}



\end{document}
